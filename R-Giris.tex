\documentclass[]{article}
\usepackage{lmodern}
\usepackage{amssymb,amsmath}
\usepackage{ifxetex,ifluatex}
\usepackage{fixltx2e} % provides \textsubscript
\ifnum 0\ifxetex 1\fi\ifluatex 1\fi=0 % if pdftex
  \usepackage[T1]{fontenc}
  \usepackage[utf8]{inputenc}
\else % if luatex or xelatex
  \ifxetex
    \usepackage{mathspec}
  \else
    \usepackage{fontspec}
  \fi
  \defaultfontfeatures{Ligatures=TeX,Scale=MatchLowercase}
\fi
% use upquote if available, for straight quotes in verbatim environments
\IfFileExists{upquote.sty}{\usepackage{upquote}}{}
% use microtype if available
\IfFileExists{microtype.sty}{%
\usepackage{microtype}
\UseMicrotypeSet[protrusion]{basicmath} % disable protrusion for tt fonts
}{}
\usepackage[margin=1in]{geometry}
\usepackage{hyperref}
\hypersetup{unicode=true,
            pdftitle={R ile analize başlarken},
            pdfauthor={Derleyen Serdar Balcı, MD, Pathologist},
            pdfborder={0 0 0},
            breaklinks=true}
\urlstyle{same}  % don't use monospace font for urls
\usepackage{color}
\usepackage{fancyvrb}
\newcommand{\VerbBar}{|}
\newcommand{\VERB}{\Verb[commandchars=\\\{\}]}
\DefineVerbatimEnvironment{Highlighting}{Verbatim}{commandchars=\\\{\}}
% Add ',fontsize=\small' for more characters per line
\usepackage{framed}
\definecolor{shadecolor}{RGB}{248,248,248}
\newenvironment{Shaded}{\begin{snugshade}}{\end{snugshade}}
\newcommand{\AlertTok}[1]{\textcolor[rgb]{0.94,0.16,0.16}{#1}}
\newcommand{\AnnotationTok}[1]{\textcolor[rgb]{0.56,0.35,0.01}{\textbf{\textit{#1}}}}
\newcommand{\AttributeTok}[1]{\textcolor[rgb]{0.77,0.63,0.00}{#1}}
\newcommand{\BaseNTok}[1]{\textcolor[rgb]{0.00,0.00,0.81}{#1}}
\newcommand{\BuiltInTok}[1]{#1}
\newcommand{\CharTok}[1]{\textcolor[rgb]{0.31,0.60,0.02}{#1}}
\newcommand{\CommentTok}[1]{\textcolor[rgb]{0.56,0.35,0.01}{\textit{#1}}}
\newcommand{\CommentVarTok}[1]{\textcolor[rgb]{0.56,0.35,0.01}{\textbf{\textit{#1}}}}
\newcommand{\ConstantTok}[1]{\textcolor[rgb]{0.00,0.00,0.00}{#1}}
\newcommand{\ControlFlowTok}[1]{\textcolor[rgb]{0.13,0.29,0.53}{\textbf{#1}}}
\newcommand{\DataTypeTok}[1]{\textcolor[rgb]{0.13,0.29,0.53}{#1}}
\newcommand{\DecValTok}[1]{\textcolor[rgb]{0.00,0.00,0.81}{#1}}
\newcommand{\DocumentationTok}[1]{\textcolor[rgb]{0.56,0.35,0.01}{\textbf{\textit{#1}}}}
\newcommand{\ErrorTok}[1]{\textcolor[rgb]{0.64,0.00,0.00}{\textbf{#1}}}
\newcommand{\ExtensionTok}[1]{#1}
\newcommand{\FloatTok}[1]{\textcolor[rgb]{0.00,0.00,0.81}{#1}}
\newcommand{\FunctionTok}[1]{\textcolor[rgb]{0.00,0.00,0.00}{#1}}
\newcommand{\ImportTok}[1]{#1}
\newcommand{\InformationTok}[1]{\textcolor[rgb]{0.56,0.35,0.01}{\textbf{\textit{#1}}}}
\newcommand{\KeywordTok}[1]{\textcolor[rgb]{0.13,0.29,0.53}{\textbf{#1}}}
\newcommand{\NormalTok}[1]{#1}
\newcommand{\OperatorTok}[1]{\textcolor[rgb]{0.81,0.36,0.00}{\textbf{#1}}}
\newcommand{\OtherTok}[1]{\textcolor[rgb]{0.56,0.35,0.01}{#1}}
\newcommand{\PreprocessorTok}[1]{\textcolor[rgb]{0.56,0.35,0.01}{\textit{#1}}}
\newcommand{\RegionMarkerTok}[1]{#1}
\newcommand{\SpecialCharTok}[1]{\textcolor[rgb]{0.00,0.00,0.00}{#1}}
\newcommand{\SpecialStringTok}[1]{\textcolor[rgb]{0.31,0.60,0.02}{#1}}
\newcommand{\StringTok}[1]{\textcolor[rgb]{0.31,0.60,0.02}{#1}}
\newcommand{\VariableTok}[1]{\textcolor[rgb]{0.00,0.00,0.00}{#1}}
\newcommand{\VerbatimStringTok}[1]{\textcolor[rgb]{0.31,0.60,0.02}{#1}}
\newcommand{\WarningTok}[1]{\textcolor[rgb]{0.56,0.35,0.01}{\textbf{\textit{#1}}}}
\usepackage{longtable,booktabs}
\usepackage{graphicx,grffile}
\makeatletter
\def\maxwidth{\ifdim\Gin@nat@width>\linewidth\linewidth\else\Gin@nat@width\fi}
\def\maxheight{\ifdim\Gin@nat@height>\textheight\textheight\else\Gin@nat@height\fi}
\makeatother
% Scale images if necessary, so that they will not overflow the page
% margins by default, and it is still possible to overwrite the defaults
% using explicit options in \includegraphics[width, height, ...]{}
\setkeys{Gin}{width=\maxwidth,height=\maxheight,keepaspectratio}
\IfFileExists{parskip.sty}{%
\usepackage{parskip}
}{% else
\setlength{\parindent}{0pt}
\setlength{\parskip}{6pt plus 2pt minus 1pt}
}
\setlength{\emergencystretch}{3em}  % prevent overfull lines
\providecommand{\tightlist}{%
  \setlength{\itemsep}{0pt}\setlength{\parskip}{0pt}}
\setcounter{secnumdepth}{0}
% Redefines (sub)paragraphs to behave more like sections
\ifx\paragraph\undefined\else
\let\oldparagraph\paragraph
\renewcommand{\paragraph}[1]{\oldparagraph{#1}\mbox{}}
\fi
\ifx\subparagraph\undefined\else
\let\oldsubparagraph\subparagraph
\renewcommand{\subparagraph}[1]{\oldsubparagraph{#1}\mbox{}}
\fi

%%% Use protect on footnotes to avoid problems with footnotes in titles
\let\rmarkdownfootnote\footnote%
\def\footnote{\protect\rmarkdownfootnote}

%%% Change title format to be more compact
\usepackage{titling}

% Create subtitle command for use in maketitle
\providecommand{\subtitle}[1]{
  \posttitle{
    \begin{center}\large#1\end{center}
    }
}

\setlength{\droptitle}{-2em}

  \title{R ile analize başlarken\footnote{Bu bir derlemedir, mümkün mertebe
  alıntılara referans vermeye çalıştım.}}
    \pretitle{\vspace{\droptitle}\centering\huge}
  \posttitle{\par}
    \author{Derleyen \href{https://sbalci.github.io/}{Serdar Balcı, MD, Pathologist}}
    \preauthor{\centering\large\emph}
  \postauthor{\par}
      \predate{\centering\large\emph}
  \postdate{\par}
    \date{2019-09-24}


\begin{document}
\maketitle

{
\setcounter{tocdepth}{5}
\tableofcontents
}
\begin{Shaded}
\begin{Highlighting}[]
\NormalTok{knitr}\OperatorTok{::}\NormalTok{opts_chunk}\OperatorTok{$}\KeywordTok{set}\NormalTok{(}
    \DataTypeTok{message =} \OtherTok{FALSE}\NormalTok{,}
    \DataTypeTok{warning =} \OtherTok{FALSE}\NormalTok{,}
    \DataTypeTok{comment =} \OtherTok{NA}\NormalTok{,}
    \DataTypeTok{include =} \OtherTok{FALSE}\NormalTok{,}
    \DataTypeTok{tidy =} \OtherTok{TRUE}
\NormalTok{)}
\end{Highlighting}
\end{Shaded}

\hypertarget{r-nerede-kullanilir}{%
\section{R nerede kullanılır}\label{r-nerede-kullanilir}}

\begin{itemize}
\tightlist
\item
  Veri düzenleme
\item
  İstatistik analiz
\item
  Web sayfası hazırlama (Statik/Dinamik)
\item
  Sunum hazırlama
\item
  Programlama
\item
  Otomatik, periodik ve tekrarlanabilir rapor hazırlama
\item
  pdf, html, ppt oluşturma
\item
  tez yazma
\item
  kitap yazma
\item
  CV oluşturma
\item
  poster hazırlama
\item
  rapor şablonu oluşturma
\item
  \ldots{}
\end{itemize}

\begin{center}\rule{0.5\linewidth}{\linethickness}\end{center}

\hypertarget{r-generation}
\end{figure}

\begin{center}\rule{0.5\linewidth}{\linethickness}\end{center}

\hypertarget{r-yukleme}{%
\section{R yükleme}\label{r-yukleme}}

\url{http://www.youtube.com/watch?v=XcBLEVknqvY}

\hypertarget{what-is-r}{%
\subsection{\texorpdfstring{\protect\includegraphics{http://img.youtube.com/vi/XcBLEVknqvY/0.jpg}}{What is R?}}\label{what-is-r}}

\hypertarget{r-project}{%
\subsection{R-project}\label{r-project}}

\url{https://cran.r-project.org/}

\begin{center}\rule{0.5\linewidth}{\linethickness}\end{center}

\hypertarget{rstudio}{%
\subsection{RStudio}\label{rstudio}}

\includegraphics{https://ismayc.github.io/talks/ness-infer/img/engine.png}

\begin{center}\rule{0.5\linewidth}{\linethickness}\end{center}

\hypertarget{rstudio-1}{%
\subsection{RStudio}\label{rstudio-1}}

\url{https://www.rstudio.com/}

\url{https://www.rstudio.com/products/rstudio/download/}

\url{https://moderndive.com/2-getting-started.html}

\begin{center}\rule{0.5\linewidth}{\linethickness}\end{center}

\href{https://buzzrbeeline.blog/2018/07/04/rstudio-anatomy/}{\includegraphics{http://www-users.york.ac.uk/~er13/RStudio\%20Anatomy.svg}}

\begin{center}\rule{0.5\linewidth}{\linethickness}\end{center}

\hypertarget{rstudio-eklentileri}{%
\subsubsection{RStudio eklentileri}\label{rstudio-eklentileri}}

\begin{itemize}
\tightlist
\item
  Discover and install useful RStudio addins
\end{itemize}

\url{https://cran.r-project.org/web/packages/addinslist/README.html}

\url{https://rstudio.github.io/rstudioaddins/}

\begin{verbatim}
devtools::install_github("rstudio/addinexamples", type = "source")
\end{verbatim}

\begin{center}\rule{0.5\linewidth}{\linethickness}\end{center}

\hypertarget{macos-icin}{%
\section{MacOS için}\label{macos-icin}}

\hypertarget{x11}{%
\subsection{X11}\label{x11}}

\url{https://www.xquartz.org/}

\hypertarget{java-os}{%
\subsection{Java OS}\label{java-os}}

\url{https://support.apple.com/kb/dl1572}

\begin{center}\rule{0.5\linewidth}{\linethickness}\end{center}

\hypertarget{r-zor-seyler-icin-kolay-kolay-seyler-icin-zor}{%
\section{R zor şeyler için kolay, kolay şeyler için
zor}\label{r-zor-seyler-icin-kolay-kolay-seyler-icin-zor}}

\begin{itemize}
\item
  \href{http://r4stats.com/articles/why-r-is-hard-to-learn/}{R makes
  easy things hard, and hard things easy}
\item
  Aynı şeyi çok fazla şekilde yapmak mümkün
\end{itemize}

R Syntax Comparison::CHEAT SHEET

\url{https://www.amelia.mn/Syntax-cheatsheet.pdf}

\begin{center}\rule{0.5\linewidth}{\linethickness}\end{center}

\#RStats --- There are always several ways to do the same thing\ldots{}
nice example on with the identity matrix by @TeaStats
https://t.co/O3GXdPiM32

--- Colin Fay 🤘 (@\_ColinFay) April 1, 2019

\begin{center}\rule{0.5\linewidth}{\linethickness}\end{center}

\hypertarget{r-paketleri}{%
\section{R paketleri}\label{r-paketleri}}

\hypertarget{neden-paketler-var}{%
\subsection{Neden paketler var}\label{neden-paketler-var}}

\href{https://ismayc.github.io/talks/ness-infer/slide_deck.html\#7}{\includegraphics{https://ismayc.github.io/talks/ness-infer/img/appstore.png}}

\begin{center}\rule{0.5\linewidth}{\linethickness}\end{center}

I love the \#rstats community.Someone is like, ``oh hey peeps, I saw a
big need for this mundane but difficult task that I infrequently do, so
I created a package that will literally scrape the last bits of peanut
butter out of the jar for you. It's called pbplyr.''What a tribe.

--- Frank Elavsky ᴰᵃᵗᵃ ᵂᶦᶻᵃʳᵈ (@Frankly\_Data) July 3, 2018

\begin{center}\rule{0.5\linewidth}{\linethickness}\end{center}

\includegraphics{https://blog.mitchelloharawild.com/blog/2018-07-11-user-2018-feature-wall_files/final.jpg}

\begin{center}\rule{0.5\linewidth}{\linethickness}\end{center}

\hypertarget{paketleri-nereden-bulabiliriz}{%
\subsection{Paketleri nereden
bulabiliriz}\label{paketleri-nereden-bulabiliriz}}

\begin{itemize}
\item
  Available CRAN Packages By Name\\
  \url{https://cran.r-project.org/web/packages/available_packages_by_name.html}
\item
  CRAN Task Views\\
  \url{https://cran.r-project.org/web/views/}
\item
  Bioconductor\\
  \url{https://www.bioconductor.org}
\item
  RecommendR\\
  \url{http://recommendr.info/}
\item
  pkgsearch\\
  CRAN package search\\
  \url{https://github.com/metacran/pkgsearch}
\item
  CRANsearcher\\
  \url{https://github.com/RhoInc/CRANsearcher}
\item
  Awesome R\\
  \url{https://awesome-r.com/}
\end{itemize}

\begin{center}\rule{0.5\linewidth}{\linethickness}\end{center}

\hypertarget{kendi-paket-evrenini-olustur}{%
\subsection{Kendi paket evrenini
oluştur}\label{kendi-paket-evrenini-olustur}}

\begin{itemize}
\tightlist
\item
  pkgverse: Build a Meta-Package Universe\\
  \url{https://cran.r-project.org/web/packages/pkgverse/index.html}
\end{itemize}

\begin{center}\rule{0.5\linewidth}{\linethickness}\end{center}

\hypertarget{r-paket-yukleme}{%
\subsection{R paket yükleme}\label{r-paket-yukleme}}

\begin{verbatim}
install.packages("tidyverse", dependencies = TRUE)
install.packages("jmv", dependencies = TRUE)
install.packages("questionr", dependencies = TRUE)
install.packages("Rcmdr", dependencies = TRUE)
install.packages("summarytools")
\end{verbatim}

\begin{center}\rule{0.5\linewidth}{\linethickness}\end{center}

\hypertarget{paket-cagirma}{%
\subsection{Paket çağırma}\label{paket-cagirma}}

\begin{Shaded}
\begin{Highlighting}[]
\CommentTok{# require(tidyverse) require(jmv) require(questionr) library(summarytools)}
\CommentTok{# library(gganimate)}
\end{Highlighting}
\end{Shaded}

\begin{center}\rule{0.5\linewidth}{\linethickness}\end{center}

\hypertarget{r-icin-yardim-bulma}
\end{figure}

\begin{center}\rule{0.5\linewidth}{\linethickness}\end{center}

\begin{itemize}
\item
  RDocumentation \url{https://www.rdocumentation.org}
\item
  R Package Documentation \url{https://rdrr.io/}
\item
  GitHub
\item
  Stackoverflow
\end{itemize}

\url{https://stackoverflow.com/}

\begin{itemize}
\tightlist
\item
  Google uygun anahtar kelime
\end{itemize}

\begin{center}\rule{0.5\linewidth}{\linethickness}\end{center}

How I use \#rstats h/t @ThePracticalDev pic.twitter.com/erRnTG0Ujr

--- Emily Bovee (@ebovee09) August 10, 2018

\begin{center}\rule{0.5\linewidth}{\linethickness}\end{center}

\includegraphics{figures/Google-package-name.png}

\begin{center}\rule{0.5\linewidth}{\linethickness}\end{center}

\includegraphics{figures/Google-start-with-R.png}

\begin{itemize}
\tightlist
\item
  Google'da ararken \texttt{{[}R{]}} yazmak da işe yarayabiliyor.
\end{itemize}

\begin{center}\rule{0.5\linewidth}{\linethickness}\end{center}

\begin{itemize}
\tightlist
\item
  searcher package 📦
\end{itemize}

\href{https://github.com/coatless/searcher}{\includegraphics{https://camo.githubusercontent.com/12f0e2d18047f1b5f36fbeb09a1d0e548236883f/68747470733a2f2f692e696d6775722e636f6d2f5a7132726736472e676966}}

\begin{center}\rule{0.5\linewidth}{\linethickness}\end{center}

\begin{itemize}
\tightlist
\item
  Awesome Cheatsheet
  \url{https://github.com/detailyang/awesome-cheatsheet}
\end{itemize}

\url{http://cran.r-project.org/doc/contrib/Baggott-refcard-v2.pdf}

\url{https://www.rstudio.com/resources/cheatsheets/}

\begin{itemize}
\tightlist
\item
  Awesome R
\end{itemize}

\url{https://github.com/qinwf/awesome-R\#readme}

\url{https://awesome-r.com/}

\begin{itemize}
\tightlist
\item
  Twitter
\end{itemize}

\url{https://twitter.com/hashtag/rstats?src=hash}

\begin{center}\rule{0.5\linewidth}{\linethickness}\end{center}

\begin{itemize}
\tightlist
\item
  Use Reproducible Examples When Asking
\end{itemize}

Got a question to ask on @SlackHQ or post on @github? No time to read
the long post on how to use reprex? Here is a 20-second gif for you to
format your R codes nicely and for others to reproduce your problem. (An
example from a talk given by @JennyBryan) \#rstat
pic.twitter.com/gpuGXpFIsX

--- ZhiYang (@zhiiiyang) October 18, 2018

\begin{itemize}
\tightlist
\item
  Keeping up to date with R news\\
  \url{https://masalmon.eu/2019/01/25/uptodate/}
\end{itemize}

\begin{center}\rule{0.5\linewidth}{\linethickness}\end{center}

\hypertarget{r-studio-ile-proje-olusturma}{%
\section{R studio ile proje
oluşturma}\label{r-studio-ile-proje-olusturma}}

\url{https://support.rstudio.com/hc/en-us/articles/200526207-Using-Projects}

\includegraphics{http://www.rstudio.com/images/docs/projects_new.png}

\begin{center}\rule{0.5\linewidth}{\linethickness}\end{center}

\hypertarget{rstudio-ile-veri-yukleme}{%
\section{RStudio ile veri yükleme}\label{rstudio-ile-veri-yukleme}}

\url{https://support.rstudio.com/hc/en-us/articles/218611977-Importing-Data-with-RStudio}

\includegraphics{https://support.rstudio.com/hc/en-us/article_attachments/206277618/data-import-overview.gif}

\begin{center}\rule{0.5\linewidth}{\linethickness}\end{center}

\hypertarget{excel}{%
\subsection{Excel}\label{excel}}

\hypertarget{spss}{%
\subsection{SPSS}\label{spss}}

\hypertarget{csv}{%
\subsection{CSV}\label{csv}}

\begin{center}\rule{0.5\linewidth}{\linethickness}\end{center}

\hypertarget{veriyi-goruntuleme}{%
\section{Veriyi görüntüleme}\label{veriyi-goruntuleme}}

Spreadsheet users using \#rstats: where's the data?\#rstats users using
spreadsheets: where's the code?

--- Leonard Kiefer (@lenkiefer) July 7, 2018

\begin{center}\rule{0.5\linewidth}{\linethickness}\end{center}

\hypertarget{veriyi-goruntuleme-1}{%
\section{Veriyi görüntüleme}\label{veriyi-goruntuleme-1}}

\begin{verbatim}
View(data)
\end{verbatim}

\begin{verbatim}
data
\end{verbatim}

\begin{verbatim}
head
\end{verbatim}

\begin{verbatim}
tail
\end{verbatim}

\begin{verbatim}
glimpse
\end{verbatim}

\begin{verbatim}
str
\end{verbatim}

\begin{verbatim}
skimr::skim()
\end{verbatim}

\begin{center}\rule{0.5\linewidth}{\linethickness}\end{center}

\hypertarget{veriyi-degistirme}{%
\section{Veriyi değiştirme}\label{veriyi-degistirme}}

\hypertarget{veriyi-kod-ile-degistirelim}{%
\subsection{Veriyi kod ile
değiştirelim}\label{veriyi-kod-ile-degistirelim}}

\hypertarget{veriyi-eklentilerle-degistirme}
\end{figure}

\begin{center}\rule{0.5\linewidth}{\linethickness}\end{center}

\hypertarget{rstudio-araciligiyla-recode}
\end{figure}

\begin{center}\rule{0.5\linewidth}{\linethickness}\end{center}

\url{https://juba.github.io/questionr/articles/recoding_addins.html}

\includegraphics{https://raw.githubusercontent.com/juba/questionr/master/resources/screenshots/irec_1.png}

\begin{center}\rule{0.5\linewidth}{\linethickness}\end{center}

\includegraphics{https://raw.githubusercontent.com/juba/questionr/master/resources/screenshots/irec_2.png}

\begin{center}\rule{0.5\linewidth}{\linethickness}\end{center}

\includegraphics{https://raw.githubusercontent.com/juba/questionr/master/resources/screenshots/irec_3.png}

\begin{center}\rule{0.5\linewidth}{\linethickness}\end{center}

\hypertarget{basit-tanimlayici-istatistikler}{%
\section{Basit tanımlayıcı
istatistikler}\label{basit-tanimlayici-istatistikler}}

\begin{verbatim}
summary()
\end{verbatim}

\begin{verbatim}
mean
\end{verbatim}

\begin{verbatim}
median
\end{verbatim}

\begin{verbatim}
min
\end{verbatim}

\begin{verbatim}
max
\end{verbatim}

\begin{verbatim}
sd
\end{verbatim}

\begin{verbatim}
table()
\end{verbatim}

\begin{center}\rule{0.5\linewidth}{\linethickness}\end{center}

\begin{Shaded}
\begin{Highlighting}[]
\KeywordTok{library}\NormalTok{(readr)}
\NormalTok{irisdata <-}\StringTok{ }\KeywordTok{read_csv}\NormalTok{(}\StringTok{"data/iris.csv"}\NormalTok{)}

\NormalTok{jmv}\OperatorTok{::}\KeywordTok{descriptives}\NormalTok{(}\DataTypeTok{data =}\NormalTok{ irisdata, }\DataTypeTok{vars =} \StringTok{"Sepal.Length"}\NormalTok{, }\DataTypeTok{splitBy =} \StringTok{"Species"}\NormalTok{, }
    \DataTypeTok{freq =} \OtherTok{TRUE}\NormalTok{, }\DataTypeTok{hist =} \OtherTok{TRUE}\NormalTok{, }\DataTypeTok{dens =} \OtherTok{TRUE}\NormalTok{, }\DataTypeTok{bar =} \OtherTok{TRUE}\NormalTok{, }\DataTypeTok{box =} \OtherTok{TRUE}\NormalTok{, }\DataTypeTok{violin =} \OtherTok{TRUE}\NormalTok{, }
    \DataTypeTok{dot =} \OtherTok{TRUE}\NormalTok{, }\DataTypeTok{mode =} \OtherTok{TRUE}\NormalTok{, }\DataTypeTok{sum =} \OtherTok{TRUE}\NormalTok{, }\DataTypeTok{sd =} \OtherTok{TRUE}\NormalTok{, }\DataTypeTok{variance =} \OtherTok{TRUE}\NormalTok{, }\DataTypeTok{range =} \OtherTok{TRUE}\NormalTok{, }
    \DataTypeTok{se =} \OtherTok{TRUE}\NormalTok{, }\DataTypeTok{skew =} \OtherTok{TRUE}\NormalTok{, }\DataTypeTok{kurt =} \OtherTok{TRUE}\NormalTok{, }\DataTypeTok{quart =} \OtherTok{TRUE}\NormalTok{, }\DataTypeTok{pcEqGr =} \OtherTok{TRUE}\NormalTok{)}
\end{Highlighting}
\end{Shaded}

\begin{verbatim}

 DESCRIPTIVES

 Descriptives                                          
 ───────────────────────────────────────────────────── 
                          Species       Sepal.Length   
 ───────────────────────────────────────────────────── 
   N                      setosa                  50   
                          versicolor              50   
                          virginica               50   
   Missing                setosa                   0   
                          versicolor               0   
                          virginica                0   
   Mean                   setosa                5.01   
                          versicolor            5.94   
                          virginica             6.59   
   Std. error mean        setosa              0.0498   
                          versicolor          0.0730   
                          virginica           0.0899   
   Median                 setosa                5.00   
                          versicolor            5.90   
                          virginica             6.50   
   Mode                   setosa                5.00   
                          versicolor            5.50   
                          virginica             6.30   
   Sum                    setosa                 250   
                          versicolor             297   
                          virginica              329   
   Standard deviation     setosa               0.352   
                          versicolor           0.516   
                          virginica            0.636   
   Variance               setosa               0.124   
                          versicolor           0.266   
                          virginica            0.404   
   Range                  setosa                1.50   
                          versicolor            2.10   
                          virginica             3.00   
   Minimum                setosa                4.30   
                          versicolor            4.90   
                          virginica             4.90   
   Maximum                setosa                5.80   
                          versicolor            7.00   
                          virginica             7.90   
   Skewness               setosa               0.120   
                          versicolor           0.105   
                          virginica            0.118   
   Std. error skewness    setosa               0.337   
                          versicolor           0.337   
                          virginica            0.337   
   Kurtosis               setosa              -0.253   
                          versicolor          -0.533   
                          virginica           0.0329   
   Std. error kurtosis    setosa               0.662   
                          versicolor           0.662   
                          virginica            0.662   
   25th percentile        setosa                4.80   
                          versicolor            5.60   
                          virginica             6.23   
   50th percentile        setosa                5.00   
                          versicolor            5.90   
                          virginica             6.50   
   75th percentile        setosa                5.20   
                          versicolor            6.30   
                          virginica             6.90   
 ───────────────────────────────────────────────────── 
\end{verbatim}

\includegraphics{R-Giris_files/figure-latex/descriptive-1.pdf}
\includegraphics{R-Giris_files/figure-latex/descriptive-2.pdf}

\begin{center}\rule{0.5\linewidth}{\linethickness}\end{center}

\begin{Shaded}
\begin{Highlighting}[]
\CommentTok{# install.packages('scatr')}

\NormalTok{scatr}\OperatorTok{::}\KeywordTok{scat}\NormalTok{(}\DataTypeTok{data =}\NormalTok{ irisdata, }\DataTypeTok{x =} \StringTok{"Sepal.Length"}\NormalTok{, }\DataTypeTok{y =} \StringTok{"Sepal.Width"}\NormalTok{, }\DataTypeTok{group =} \StringTok{"Species"}\NormalTok{, }
    \DataTypeTok{marg =} \StringTok{"dens"}\NormalTok{, }\DataTypeTok{line =} \StringTok{"linear"}\NormalTok{, }\DataTypeTok{se =} \OtherTok{TRUE}\NormalTok{)}
\end{Highlighting}
\end{Shaded}

\includegraphics{R-Giris_files/figure-latex/scatter-1.pdf}

\begin{center}\rule{0.5\linewidth}{\linethickness}\end{center}

\hypertarget{summarytools}{%
\subsection{summarytools}\label{summarytools}}

\url{https://cran.r-project.org/web/packages/summarytools/vignettes/Introduction.html}

\begin{Shaded}
\begin{Highlighting}[]
\KeywordTok{library}\NormalTok{(summarytools)}
\NormalTok{summarytools}\OperatorTok{::}\KeywordTok{freq}\NormalTok{(iris}\OperatorTok{$}\NormalTok{Species, }\DataTypeTok{style =} \StringTok{"rmarkdown"}\NormalTok{)}
\end{Highlighting}
\end{Shaded}

\hypertarget{frequencies}{%
\subsubsection{Frequencies}\label{frequencies}}

\hypertarget{irisspecies}{%
\paragraph{iris\$Species}\label{irisspecies}}

\textbf{Type:} Factor

\begin{longtable}[]{@{}rrrrrr@{}}
\toprule
~ & Freq & \% Valid & \% Valid Cum. & \% Total & \% Total
Cum.\tabularnewline
\midrule
\endhead
\textbf{setosa} & 50 & 33.33 & 33.33 & 33.33 & 33.33\tabularnewline
\textbf{versicolor} & 50 & 33.33 & 66.67 & 33.33 & 66.67\tabularnewline
\textbf{virginica} & 50 & 33.33 & 100.00 & 33.33 & 100.00\tabularnewline
\textbf{\textless NA\textgreater{}} & 0 & & & 0.00 &
100.00\tabularnewline
\textbf{Total} & 150 & 100.00 & 100.00 & 100.00 & 100.00\tabularnewline
\bottomrule
\end{longtable}

\begin{center}\rule{0.5\linewidth}{\linethickness}\end{center}

\begin{Shaded}
\begin{Highlighting}[]
\NormalTok{summarytools}\OperatorTok{::}\KeywordTok{freq}\NormalTok{(iris}\OperatorTok{$}\NormalTok{Species, }\DataTypeTok{report.nas =} \OtherTok{FALSE}\NormalTok{, }\DataTypeTok{style =} \StringTok{"rmarkdown"}\NormalTok{, }\DataTypeTok{headings =} \OtherTok{FALSE}\NormalTok{)}
\end{Highlighting}
\end{Shaded}

\begin{longtable}[]{@{}rrrr@{}}
\toprule
~ & Freq & \% & \% Cum.\tabularnewline
\midrule
\endhead
\textbf{setosa} & 50 & 33.33 & 33.33\tabularnewline
\textbf{versicolor} & 50 & 33.33 & 66.67\tabularnewline
\textbf{virginica} & 50 & 33.33 & 100.00\tabularnewline
\textbf{Total} & 150 & 100.00 & 100.00\tabularnewline
\bottomrule
\end{longtable}

\begin{Shaded}
\begin{Highlighting}[]
\KeywordTok{with}\NormalTok{(tobacco, }\KeywordTok{print}\NormalTok{(}\KeywordTok{ctable}\NormalTok{(smoker, diseased), }\DataTypeTok{method =} \StringTok{"render"}\NormalTok{))}
\end{Highlighting}
\end{Shaded}

Cross-Tabulation, Row Proportions

smoker * diseased

Data Frame: tobacco

diseased

smoker

Yes

No

Total

Yes

125

(

41.9\%

)

173

(

58.1\%

)

298

(

100.0\%

)

No

99

(

14.1\%

)

603

(

85.9\%

)

702

(

100.0\%

)

Total

224

(

22.4\%

)

776

(

77.6\%

)

1000

(

100.0\%

)

Generated by summarytools 0.9.4 (R version 3.6.0)2019-09-24

\begin{Shaded}
\begin{Highlighting}[]
\KeywordTok{with}\NormalTok{(tobacco, }\KeywordTok{print}\NormalTok{(}\KeywordTok{ctable}\NormalTok{(smoker, diseased, }\DataTypeTok{prop =} \StringTok{"n"}\NormalTok{, }\DataTypeTok{totals =} \OtherTok{FALSE}\NormalTok{), }\DataTypeTok{omit.headings =} \OtherTok{TRUE}\NormalTok{, }
    \DataTypeTok{method =} \StringTok{"render"}\NormalTok{))}
\end{Highlighting}
\end{Shaded}

Cross-Tabulation

smoker * diseased

Data Frame: tobacco

diseased

smoker

Yes

No

Yes

{125}

{173}

No

{99}

{603}

Generated by summarytools 0.9.4 (R version 3.6.0)2019-09-24

\begin{center}\rule{0.5\linewidth}{\linethickness}\end{center}

\begin{Shaded}
\begin{Highlighting}[]
\NormalTok{summarytools}\OperatorTok{::}\KeywordTok{descr}\NormalTok{(iris, }\DataTypeTok{style =} \StringTok{"rmarkdown"}\NormalTok{)}
\end{Highlighting}
\end{Shaded}

\hypertarget{descriptive-statistics}{%
\subsubsection{Descriptive Statistics}\label{descriptive-statistics}}

\hypertarget{iris}{%
\paragraph{iris}\label{iris}}

\textbf{N:} 150

\begin{longtable}[]{@{}rrrrr@{}}
\toprule
~ & Petal.Length & Petal.Width & Sepal.Length &
Sepal.Width\tabularnewline
\midrule
\endhead
\textbf{Mean} & 3.76 & 1.20 & 5.84 & 3.06\tabularnewline
\textbf{Std.Dev} & 1.77 & 0.76 & 0.83 & 0.44\tabularnewline
\textbf{Min} & 1.00 & 0.10 & 4.30 & 2.00\tabularnewline
\textbf{Q1} & 1.60 & 0.30 & 5.10 & 2.80\tabularnewline
\textbf{Median} & 4.35 & 1.30 & 5.80 & 3.00\tabularnewline
\textbf{Q3} & 5.10 & 1.80 & 6.40 & 3.30\tabularnewline
\textbf{Max} & 6.90 & 2.50 & 7.90 & 4.40\tabularnewline
\textbf{MAD} & 1.85 & 1.04 & 1.04 & 0.44\tabularnewline
\textbf{IQR} & 3.50 & 1.50 & 1.30 & 0.50\tabularnewline
\textbf{CV} & 0.47 & 0.64 & 0.14 & 0.14\tabularnewline
\textbf{Skewness} & -0.27 & -0.10 & 0.31 & 0.31\tabularnewline
\textbf{SE.Skewness} & 0.20 & 0.20 & 0.20 & 0.20\tabularnewline
\textbf{Kurtosis} & -1.42 & -1.36 & -0.61 & 0.14\tabularnewline
\textbf{N.Valid} & 150.00 & 150.00 & 150.00 & 150.00\tabularnewline
\textbf{Pct.Valid} & 100.00 & 100.00 & 100.00 & 100.00\tabularnewline
\bottomrule
\end{longtable}

\begin{center}\rule{0.5\linewidth}{\linethickness}\end{center}

\begin{Shaded}
\begin{Highlighting}[]
\KeywordTok{descr}\NormalTok{(iris, }\DataTypeTok{stats =} \KeywordTok{c}\NormalTok{(}\StringTok{"mean"}\NormalTok{, }\StringTok{"sd"}\NormalTok{, }\StringTok{"min"}\NormalTok{, }\StringTok{"med"}\NormalTok{, }\StringTok{"max"}\NormalTok{), }\DataTypeTok{transpose =} \OtherTok{TRUE}\NormalTok{, }
    \DataTypeTok{headings =} \OtherTok{FALSE}\NormalTok{, }\DataTypeTok{style =} \StringTok{"rmarkdown"}\NormalTok{)}
\end{Highlighting}
\end{Shaded}

\begin{longtable}[]{@{}rrrrrr@{}}
\toprule
~ & Mean & Std.Dev & Min & Median & Max\tabularnewline
\midrule
\endhead
\textbf{Petal.Length} & 3.76 & 1.77 & 1.00 & 4.35 & 6.90\tabularnewline
\textbf{Petal.Width} & 1.20 & 0.76 & 0.10 & 1.30 & 2.50\tabularnewline
\textbf{Sepal.Length} & 5.84 & 0.83 & 4.30 & 5.80 & 7.90\tabularnewline
\textbf{Sepal.Width} & 3.06 & 0.44 & 2.00 & 3.00 & 4.40\tabularnewline
\bottomrule
\end{longtable}

\begin{center}\rule{0.5\linewidth}{\linethickness}\end{center}

\begin{Shaded}
\begin{Highlighting}[]
\CommentTok{# view(dfSummary(iris))}
\end{Highlighting}
\end{Shaded}

\includegraphics{figures/dfsummary.png}

\begin{center}\rule{0.5\linewidth}{\linethickness}\end{center}

\begin{Shaded}
\begin{Highlighting}[]
\KeywordTok{dfSummary}\NormalTok{(tobacco, }\DataTypeTok{plain.ascii =} \OtherTok{FALSE}\NormalTok{, }\DataTypeTok{style =} \StringTok{"grid"}\NormalTok{)}
\end{Highlighting}
\end{Shaded}

\hypertarget{data-frame-summary}{%
\subsubsection{Data Frame Summary}\label{data-frame-summary}}

\hypertarget{tobacco}{%
\paragraph{tobacco}\label{tobacco}}

\textbf{Dimensions:} 1000 x 9\\
\textbf{Duplicates:} 2

\begin{longtable}[]{@{}lllllll@{}}
\toprule
\begin{minipage}[b]{0.03\columnwidth}\raggedright
No\strut
\end{minipage} & \begin{minipage}[b]{0.11\columnwidth}\raggedright
Variable\strut
\end{minipage} & \begin{minipage}[b]{0.18\columnwidth}\raggedright
Stats / Values\strut
\end{minipage} & \begin{minipage}[b]{0.15\columnwidth}\raggedright
Freqs (\% of Valid)\strut
\end{minipage} & \begin{minipage}[b]{0.21\columnwidth}\raggedright
Graph\strut
\end{minipage} & \begin{minipage}[b]{0.07\columnwidth}\raggedright
Valid\strut
\end{minipage} & \begin{minipage}[b]{0.07\columnwidth}\raggedright
Missing\strut
\end{minipage}\tabularnewline
\midrule
\endhead
\begin{minipage}[t]{0.03\columnwidth}\raggedright
1\strut
\end{minipage} & \begin{minipage}[t]{0.11\columnwidth}\raggedright
gender\\
{[}factor{]}\strut
\end{minipage} & \begin{minipage}[t]{0.18\columnwidth}\raggedright
1. F\\
2. M\strut
\end{minipage} & \begin{minipage}[t]{0.15\columnwidth}\raggedright
489 (50.0\%)\\
489 (50.0\%)\strut
\end{minipage} & \begin{minipage}[t]{0.21\columnwidth}\raggedright
IIIIIIIIII\\
IIIIIIIIII\strut
\end{minipage} & \begin{minipage}[t]{0.07\columnwidth}\raggedright
978\\
(97.8\%)\strut
\end{minipage} & \begin{minipage}[t]{0.07\columnwidth}\raggedright
22\\
(2.2\%)\strut
\end{minipage}\tabularnewline
\begin{minipage}[t]{0.03\columnwidth}\raggedright
2\strut
\end{minipage} & \begin{minipage}[t]{0.11\columnwidth}\raggedright
age\\
{[}numeric{]}\strut
\end{minipage} & \begin{minipage}[t]{0.18\columnwidth}\raggedright
Mean (sd) : 49.6 (18.3)\\
min \textless{} med \textless{} max:\\
18 \textless{} 50 \textless{} 80\\
IQR (CV) : 32 (0.4)\strut
\end{minipage} & \begin{minipage}[t]{0.15\columnwidth}\raggedright
63 distinct values\strut
\end{minipage} & \begin{minipage}[t]{0.21\columnwidth}\raggedright
. ~~~~. ~~~~. . . :\\
: : : : : . : : : :\\
: : : : : : : : : :\\
: : : : : : : : : :\\
: : : : : : : : : :\strut
\end{minipage} & \begin{minipage}[t]{0.07\columnwidth}\raggedright
975\\
(97.5\%)\strut
\end{minipage} & \begin{minipage}[t]{0.07\columnwidth}\raggedright
25\\
(2.5\%)\strut
\end{minipage}\tabularnewline
\begin{minipage}[t]{0.03\columnwidth}\raggedright
3\strut
\end{minipage} & \begin{minipage}[t]{0.11\columnwidth}\raggedright
age.gr\\
{[}factor{]}\strut
\end{minipage} & \begin{minipage}[t]{0.18\columnwidth}\raggedright
1. 18-34\\
2. 35-50\\
3. 51-70\\
4. 71 +\strut
\end{minipage} & \begin{minipage}[t]{0.15\columnwidth}\raggedright
258 (26.5\%)\\
241 (24.7\%)\\
317 (32.5\%)\\
159 (16.3\%)\strut
\end{minipage} & \begin{minipage}[t]{0.21\columnwidth}\raggedright
IIIII\\
IIII\\
IIIIII\\
III\strut
\end{minipage} & \begin{minipage}[t]{0.07\columnwidth}\raggedright
975\\
(97.5\%)\strut
\end{minipage} & \begin{minipage}[t]{0.07\columnwidth}\raggedright
25\\
(2.5\%)\strut
\end{minipage}\tabularnewline
\begin{minipage}[t]{0.03\columnwidth}\raggedright
4\strut
\end{minipage} & \begin{minipage}[t]{0.11\columnwidth}\raggedright
BMI\\
{[}numeric{]}\strut
\end{minipage} & \begin{minipage}[t]{0.18\columnwidth}\raggedright
Mean (sd) : 25.7 (4.5)\\
min \textless{} med \textless{} max:\\
8.8 \textless{} 25.6 \textless{} 39.4\\
IQR (CV) : 5.7 (0.2)\strut
\end{minipage} & \begin{minipage}[t]{0.15\columnwidth}\raggedright
974 distinct values\strut
\end{minipage} & \begin{minipage}[t]{0.21\columnwidth}\raggedright
~~~~~~~~~~:\\
\hspace*{0.333em}\hspace*{0.333em}\hspace*{0.333em}\hspace*{0.333em}\hspace*{0.333em}\hspace*{0.333em}\hspace*{0.333em}\hspace*{0.333em}:
: :\\
\hspace*{0.333em}\hspace*{0.333em}\hspace*{0.333em}\hspace*{0.333em}\hspace*{0.333em}\hspace*{0.333em}\hspace*{0.333em}\hspace*{0.333em}:
: :\\
\hspace*{0.333em}\hspace*{0.333em}\hspace*{0.333em}\hspace*{0.333em}\hspace*{0.333em}\hspace*{0.333em}:
: : : :\\
\hspace*{0.333em}\hspace*{0.333em}\hspace*{0.333em}\hspace*{0.333em}. :
: : : : .\strut
\end{minipage} & \begin{minipage}[t]{0.07\columnwidth}\raggedright
974\\
(97.4\%)\strut
\end{minipage} & \begin{minipage}[t]{0.07\columnwidth}\raggedright
26\\
(2.6\%)\strut
\end{minipage}\tabularnewline
\begin{minipage}[t]{0.03\columnwidth}\raggedright
5\strut
\end{minipage} & \begin{minipage}[t]{0.11\columnwidth}\raggedright
smoker\\
{[}factor{]}\strut
\end{minipage} & \begin{minipage}[t]{0.18\columnwidth}\raggedright
1. Yes\\
2. No\strut
\end{minipage} & \begin{minipage}[t]{0.15\columnwidth}\raggedright
298 (29.8\%)\\
702 (70.2\%)\strut
\end{minipage} & \begin{minipage}[t]{0.21\columnwidth}\raggedright
IIIII\\
IIIIIIIIIIIIII\strut
\end{minipage} & \begin{minipage}[t]{0.07\columnwidth}\raggedright
1000\\
(100\%)\strut
\end{minipage} & \begin{minipage}[t]{0.07\columnwidth}\raggedright
0\\
(0\%)\strut
\end{minipage}\tabularnewline
\begin{minipage}[t]{0.03\columnwidth}\raggedright
6\strut
\end{minipage} & \begin{minipage}[t]{0.11\columnwidth}\raggedright
cigs.per.day\\
{[}numeric{]}\strut
\end{minipage} & \begin{minipage}[t]{0.18\columnwidth}\raggedright
Mean (sd) : 6.8 (11.9)\\
min \textless{} med \textless{} max:\\
0 \textless{} 0 \textless{} 40\\
IQR (CV) : 11 (1.8)\strut
\end{minipage} & \begin{minipage}[t]{0.15\columnwidth}\raggedright
37 distinct values\strut
\end{minipage} & \begin{minipage}[t]{0.21\columnwidth}\raggedright
:\\
:\\
:\\
:\\
: ~~. . . . . .\strut
\end{minipage} & \begin{minipage}[t]{0.07\columnwidth}\raggedright
965\\
(96.5\%)\strut
\end{minipage} & \begin{minipage}[t]{0.07\columnwidth}\raggedright
35\\
(3.5\%)\strut
\end{minipage}\tabularnewline
\begin{minipage}[t]{0.03\columnwidth}\raggedright
7\strut
\end{minipage} & \begin{minipage}[t]{0.11\columnwidth}\raggedright
diseased\\
{[}factor{]}\strut
\end{minipage} & \begin{minipage}[t]{0.18\columnwidth}\raggedright
1. Yes\\
2. No\strut
\end{minipage} & \begin{minipage}[t]{0.15\columnwidth}\raggedright
224 (22.4\%)\\
776 (77.6\%)\strut
\end{minipage} & \begin{minipage}[t]{0.21\columnwidth}\raggedright
IIII\\
IIIIIIIIIIIIIII\strut
\end{minipage} & \begin{minipage}[t]{0.07\columnwidth}\raggedright
1000\\
(100\%)\strut
\end{minipage} & \begin{minipage}[t]{0.07\columnwidth}\raggedright
0\\
(0\%)\strut
\end{minipage}\tabularnewline
\begin{minipage}[t]{0.03\columnwidth}\raggedright
8\strut
\end{minipage} & \begin{minipage}[t]{0.11\columnwidth}\raggedright
disease\\
{[}character{]}\strut
\end{minipage} & \begin{minipage}[t]{0.18\columnwidth}\raggedright
1. Hypertension\\
2. Cancer\\
3. Cholesterol\\
4. Heart\\
5. Pulmonary\\
6. Musculoskeletal\\
7. Diabetes\\
8. Hearing\\
9. Digestive\\
10. Hypotension\\
{[} 3 others {]}\strut
\end{minipage} & \begin{minipage}[t]{0.15\columnwidth}\raggedright
36 (16.2\%)\\
34 (15.3\%)\\
21 ( 9.5\%)\\
20 ( 9.0\%)\\
20 ( 9.0\%)\\
19 ( 8.6\%)\\
14 ( 6.3\%)\\
14 ( 6.3\%)\\
12 ( 5.4\%)\\
11 ( 5.0\%)\\
21 ( 9.5\%)\strut
\end{minipage} & \begin{minipage}[t]{0.21\columnwidth}\raggedright
III\\
III\\
I\\
I\\
I\\
I\\
I\\
I\\
I\\
~\\
I\strut
\end{minipage} & \begin{minipage}[t]{0.07\columnwidth}\raggedright
222\\
(22.2\%)\strut
\end{minipage} & \begin{minipage}[t]{0.07\columnwidth}\raggedright
778\\
(77.8\%)\strut
\end{minipage}\tabularnewline
\begin{minipage}[t]{0.03\columnwidth}\raggedright
9\strut
\end{minipage} & \begin{minipage}[t]{0.11\columnwidth}\raggedright
samp.wgts\\
{[}numeric{]}\strut
\end{minipage} & \begin{minipage}[t]{0.18\columnwidth}\raggedright
Mean (sd) : 1 (0.1)\\
min \textless{} med \textless{} max:\\
0.9 \textless{} 1 \textless{} 1.1\\
IQR (CV) : 0.2 (0.1)\strut
\end{minipage} & \begin{minipage}[t]{0.15\columnwidth}\raggedright
0.86!: 267 (26.7\%)\\
1.04!: 249 (24.9\%)\\
1.05!: 324 (32.4\%)\\
1.06!: 160 (16.0\%)\\
! rounded\strut
\end{minipage} & \begin{minipage}[t]{0.21\columnwidth}\raggedright
IIIII\\
IIII\\
IIIIII\\
III\\
~\\
\strut
\end{minipage} & \begin{minipage}[t]{0.07\columnwidth}\raggedright
1000\\
(100\%)\strut
\end{minipage} & \begin{minipage}[t]{0.07\columnwidth}\raggedright
0\\
(0\%)\strut
\end{minipage}\tabularnewline
\bottomrule
\end{longtable}

\begin{center}\rule{0.5\linewidth}{\linethickness}\end{center}

\begin{Shaded}
\begin{Highlighting}[]
\CommentTok{# First save the results}

\NormalTok{iris_stats_by_species <-}\StringTok{ }\KeywordTok{by}\NormalTok{(}\DataTypeTok{data =}\NormalTok{ iris, }\DataTypeTok{INDICES =}\NormalTok{ iris}\OperatorTok{$}\NormalTok{Species, }\DataTypeTok{FUN =}\NormalTok{ descr, }
    \DataTypeTok{stats =} \KeywordTok{c}\NormalTok{(}\StringTok{"mean"}\NormalTok{, }\StringTok{"sd"}\NormalTok{, }\StringTok{"min"}\NormalTok{, }\StringTok{"med"}\NormalTok{, }\StringTok{"max"}\NormalTok{), }\DataTypeTok{transpose =} \OtherTok{TRUE}\NormalTok{)}

\CommentTok{# Then use view(), like so:}

\KeywordTok{view}\NormalTok{(iris_stats_by_species, }\DataTypeTok{method =} \StringTok{"pander"}\NormalTok{, }\DataTypeTok{style =} \StringTok{"rmarkdown"}\NormalTok{)}
\end{Highlighting}
\end{Shaded}

\hypertarget{descriptive-statistics-1}{%
\subsubsection{Descriptive Statistics}\label{descriptive-statistics-1}}

\hypertarget{iris-1}{%
\paragraph{iris}\label{iris-1}}

\textbf{Group:} Species = setosa\\
\textbf{N:} 50

\begin{longtable}[]{@{}rrrrrr@{}}
\toprule
~ & Mean & Std.Dev & Min & Median & Max\tabularnewline
\midrule
\endhead
\textbf{Petal.Length} & 1.46 & 0.17 & 1.00 & 1.50 & 1.90\tabularnewline
\textbf{Petal.Width} & 0.25 & 0.11 & 0.10 & 0.20 & 0.60\tabularnewline
\textbf{Sepal.Length} & 5.01 & 0.35 & 4.30 & 5.00 & 5.80\tabularnewline
\textbf{Sepal.Width} & 3.43 & 0.38 & 2.30 & 3.40 & 4.40\tabularnewline
\bottomrule
\end{longtable}

\textbf{Group:} Species = versicolor\\
\textbf{N:} 50

\begin{longtable}[]{@{}rrrrrr@{}}
\toprule
~ & Mean & Std.Dev & Min & Median & Max\tabularnewline
\midrule
\endhead
\textbf{Petal.Length} & 4.26 & 0.47 & 3.00 & 4.35 & 5.10\tabularnewline
\textbf{Petal.Width} & 1.33 & 0.20 & 1.00 & 1.30 & 1.80\tabularnewline
\textbf{Sepal.Length} & 5.94 & 0.52 & 4.90 & 5.90 & 7.00\tabularnewline
\textbf{Sepal.Width} & 2.77 & 0.31 & 2.00 & 2.80 & 3.40\tabularnewline
\bottomrule
\end{longtable}

\textbf{Group:} Species = virginica\\
\textbf{N:} 50

\begin{longtable}[]{@{}rrrrrr@{}}
\toprule
~ & Mean & Std.Dev & Min & Median & Max\tabularnewline
\midrule
\endhead
\textbf{Petal.Length} & 5.55 & 0.55 & 4.50 & 5.55 & 6.90\tabularnewline
\textbf{Petal.Width} & 2.03 & 0.27 & 1.40 & 2.00 & 2.50\tabularnewline
\textbf{Sepal.Length} & 6.59 & 0.64 & 4.90 & 6.50 & 7.90\tabularnewline
\textbf{Sepal.Width} & 2.97 & 0.32 & 2.20 & 3.00 & 3.80\tabularnewline
\bottomrule
\end{longtable}

\begin{center}\rule{0.5\linewidth}{\linethickness}\end{center}

\begin{Shaded}
\begin{Highlighting}[]
\CommentTok{# view(iris_stats_by_species)}
\end{Highlighting}
\end{Shaded}

\includegraphics{figures/DescriptiveStatistics.png}

\begin{center}\rule{0.5\linewidth}{\linethickness}\end{center}

\begin{Shaded}
\begin{Highlighting}[]
\KeywordTok{data}\NormalTok{(tobacco)  }\CommentTok{# tobacco is an example dataframe included in the package}
\NormalTok{BMI_by_age <-}\StringTok{ }\KeywordTok{with}\NormalTok{(tobacco, }\KeywordTok{by}\NormalTok{(BMI, age.gr, descr, }\DataTypeTok{stats =} \KeywordTok{c}\NormalTok{(}\StringTok{"mean"}\NormalTok{, }\StringTok{"sd"}\NormalTok{, }\StringTok{"min"}\NormalTok{, }
    \StringTok{"med"}\NormalTok{, }\StringTok{"max"}\NormalTok{)))}
\KeywordTok{view}\NormalTok{(BMI_by_age, }\StringTok{"pander"}\NormalTok{, }\DataTypeTok{style =} \StringTok{"rmarkdown"}\NormalTok{)}
\end{Highlighting}
\end{Shaded}

\hypertarget{descriptive-statistics-2}{%
\subsubsection{Descriptive Statistics}\label{descriptive-statistics-2}}

\hypertarget{bmi-by-age.gr}{%
\paragraph{BMI by age.gr}\label{bmi-by-age.gr}}

\textbf{Data Frame:} tobacco\\
\textbf{N:} 258

\begin{longtable}[]{@{}rrrrr@{}}
\toprule
~ & 18-34 & 35-50 & 51-70 & 71 +\tabularnewline
\midrule
\endhead
\textbf{Mean} & 23.84 & 25.11 & 26.91 & 27.45\tabularnewline
\textbf{Std.Dev} & 4.23 & 4.34 & 4.26 & 4.37\tabularnewline
\textbf{Min} & 8.83 & 10.35 & 9.01 & 16.36\tabularnewline
\textbf{Median} & 24.04 & 25.11 & 26.77 & 27.52\tabularnewline
\textbf{Max} & 34.84 & 39.44 & 39.21 & 38.37\tabularnewline
\bottomrule
\end{longtable}

\begin{center}\rule{0.5\linewidth}{\linethickness}\end{center}

\begin{Shaded}
\begin{Highlighting}[]
\NormalTok{BMI_by_age <-}\StringTok{ }\KeywordTok{with}\NormalTok{(tobacco, }\KeywordTok{by}\NormalTok{(BMI, age.gr, descr, }\DataTypeTok{transpose =} \OtherTok{TRUE}\NormalTok{, }\DataTypeTok{stats =} \KeywordTok{c}\NormalTok{(}\StringTok{"mean"}\NormalTok{, }
    \StringTok{"sd"}\NormalTok{, }\StringTok{"min"}\NormalTok{, }\StringTok{"med"}\NormalTok{, }\StringTok{"max"}\NormalTok{)))}

\KeywordTok{view}\NormalTok{(BMI_by_age, }\StringTok{"pander"}\NormalTok{, }\DataTypeTok{style =} \StringTok{"rmarkdown"}\NormalTok{, }\DataTypeTok{omit.headings =} \OtherTok{TRUE}\NormalTok{)}
\end{Highlighting}
\end{Shaded}

\hypertarget{descriptive-statistics-3}{%
\subsubsection{Descriptive Statistics}\label{descriptive-statistics-3}}

\hypertarget{bmi-by-age.gr-1}{%
\paragraph{BMI by age.gr}\label{bmi-by-age.gr-1}}

\textbf{Data Frame:} tobacco\\
\textbf{N:} 258

\begin{longtable}[]{@{}rrrrrr@{}}
\toprule
~ & Mean & Std.Dev & Min & Median & Max\tabularnewline
\midrule
\endhead
\textbf{18-34} & 23.84 & 4.23 & 8.83 & 24.04 & 34.84\tabularnewline
\textbf{35-50} & 25.11 & 4.34 & 10.35 & 25.11 & 39.44\tabularnewline
\textbf{51-70} & 26.91 & 4.26 & 9.01 & 26.77 & 39.21\tabularnewline
\textbf{71 +} & 27.45 & 4.37 & 16.36 & 27.52 & 38.37\tabularnewline
\bottomrule
\end{longtable}

\begin{center}\rule{0.5\linewidth}{\linethickness}\end{center}

\begin{Shaded}
\begin{Highlighting}[]
\NormalTok{tobacco_subset <-}\StringTok{ }\NormalTok{tobacco[, }\KeywordTok{c}\NormalTok{(}\StringTok{"gender"}\NormalTok{, }\StringTok{"age.gr"}\NormalTok{, }\StringTok{"smoker"}\NormalTok{)]}
\NormalTok{freq_tables <-}\StringTok{ }\KeywordTok{lapply}\NormalTok{(tobacco_subset, freq)}

\CommentTok{# view(freq_tables, footnote = NA, file = 'freq-tables.html')}
\end{Highlighting}
\end{Shaded}

\begin{center}\rule{0.5\linewidth}{\linethickness}\end{center}

\begin{Shaded}
\begin{Highlighting}[]
\KeywordTok{what.is}\NormalTok{(iris)}
\end{Highlighting}
\end{Shaded}

\$properties property value 1 class data.frame 2 typeof list 3 mode list
4 storage.mode list 5 dim 150 x 5 6 length 5 7 is.object TRUE 8
object.type S3 9 object.size 7256 Bytes

\$attributes.lengths names class row.names 5 1 150

\$extensive.is {[}1{]} ``is.data.frame'' ``is.list'' ``is.object''
``is.recursive'' {[}5{]} ``is.unsorted''

\begin{center}\rule{0.5\linewidth}{\linethickness}\end{center}

\begin{center}\rule{0.5\linewidth}{\linethickness}\end{center}

\begin{center}\rule{0.5\linewidth}{\linethickness}\end{center}

\hypertarget{skimr}{%
\subsection{skimr}\label{skimr}}

\begin{verbatim}
library(skimr)
skim(df)
\end{verbatim}

\begin{center}\rule{0.5\linewidth}{\linethickness}\end{center}

\hypertarget{dataexplorer}{%
\subsection{DataExplorer}\label{dataexplorer}}

\begin{verbatim}
library(DataExplorer)
DataExplorer::create_report(df)
\end{verbatim}

\href{https://www.littlemissdata.com/blog/simple-eda}{\includegraphics{https://static1.squarespace.com/static/58eef8846a4963e429687a4d/t/5bdfc2fb4d7a9c04ee50b7aa/1541391160702/dataExplorerGifLg.gif?format=1500w}}

\begin{center}\rule{0.5\linewidth}{\linethickness}\end{center}

\hypertarget{inspectdf}{%
\subsection{inspectdf}\label{inspectdf}}

\url{https://github.com/alastairrushworth/inspectdf}

\begin{center}\rule{0.5\linewidth}{\linethickness}\end{center}

\hypertarget{grafikler}{%
\subsection{Grafikler}\label{grafikler}}

\begin{center}\rule{0.5\linewidth}{\linethickness}\end{center}

\begin{center}\rule{0.5\linewidth}{\linethickness}\end{center}

\begin{center}\rule{0.5\linewidth}{\linethickness}\end{center}

\begin{verbatim}
descr(tobacco, style = 'rmarkdown')

print(descr(tobacco), method = 'render', table.classes = 'st-small')

dfSummary(tobacco, style = 'grid', plain.ascii = FALSE)

print(dfSummary(tobacco, graph.magnif = 0.75), method = 'render')
\end{verbatim}

\begin{center}\rule{0.5\linewidth}{\linethickness}\end{center}

Here, building up a \#ggplot2 as slowly as possible, \#rstats.
Incremental adjustments. \#rstatsteachingideas
pic.twitter.com/nUulQl8bPh

--- Gina Reynolds (@EvaMaeRey) August 13, 2018

\begin{center}\rule{0.5\linewidth}{\linethickness}\end{center}

\href{https://github.com/dreamRs/esquisse}{\includegraphics{https://raw.githubusercontent.com/dreamRs/esquisse/master/man/figures/esquisse.gif}}

Dreaming of a fancy \#Rstats \#ggplot \#dataviz but still scared of
typing \#code? @\_pvictorr esquisse package has you covered
https://t.co/1vIDXcVAAF pic.twitter.com/RlTkptnrNv

--- Radoslaw Panczak (@RPanczak) October 2, 2018

\begin{center}\rule{0.5\linewidth}{\linethickness}\end{center}

\hypertarget{bazi-arayuzler}{%
\section{Bazı arayüzler}\label{bazi-arayuzler}}

\hypertarget{rcmdr}{%
\subsection{Rcmdr}\label{rcmdr}}

\begin{verbatim}
library(Rcmdr)

Rcmdr::Commander()
\end{verbatim}

\begin{itemize}
\tightlist
\item
  A Comparative Review of the R Commander GUI for R
\end{itemize}

\url{http://r4stats.com/articles/software-reviews/r-commander/}

\begin{center}\rule{0.5\linewidth}{\linethickness}\end{center}

\hypertarget{jamovi}{%
\subsection{jamovi}\label{jamovi}}

\url{https://www.jamovi.org/}

\begin{figure}
\centering
\includegraphics{https://www.jamovi.org/}
\caption{\includegraphics{https://www.jamovi.org/assets/main-screenshot.png}}
\end{figure}

\url{https://blog.jamovi.org/2018/07/30/rj.html}

\begin{figure}
\centering
\includegraphics{https://blog.jamovi.org/2018/07/30/rj.html}
\caption{\includegraphics{https://blog.jamovi.org/assets/images/rj.png}}
\end{figure}

\begin{center}\rule{0.5\linewidth}{\linethickness}\end{center}

\hypertarget{r-nereden-ogrenilir}{%
\section{R nereden öğrenilir}\label{r-nereden-ogrenilir}}

\url{https://sbalci.github.io/MyRCodesForDataAnalysis/WhereToLearnR.nb.html}

\begin{center}\rule{0.5\linewidth}{\linethickness}\end{center}

\hypertarget{sonraki-konular}{%
\section{Sonraki Konular}\label{sonraki-konular}}

\begin{itemize}
\tightlist
\item
  RStudio ile GitHub kullanımı
\item
  R Markdown ve R Notebook ile tekrarlanabilir rapor
\item
  Hipotez testleri
\end{itemize}

\begin{center}\rule{0.5\linewidth}{\linethickness}\end{center}

\hypertarget{geri-bildirim}{%
\section{Geri Bildirim}\label{geri-bildirim}}

\begin{itemize}
\tightlist
\item
  Geri bildirim için tıklayınız:
  \emph{\href{https://goo.gl/forms/YjGZ5DHgtPlR1RnB3}{Geri bildirim
  formu}}
\end{itemize}

\begin{center}\rule{0.5\linewidth}{\linethickness}\end{center}

\hypertarget{disqus_thread}{}

Please enable JavaScript to view the comments powered by Disqus.

\begin{center}\rule{0.5\linewidth}{\linethickness}\end{center}

\begin{verbatim}
# Save Final Data

saved data after analysis to `Data-After-Analysis.xlsx`.

saveRDS(mydata, "Data-After-Analysis.rds")

writexl::write_xlsx(mydata, "Data-After-Analysis.xlsx")

file.info("Data-After-Analysis.xlsx")$ctime
\end{verbatim}

\begin{center}\rule{0.5\linewidth}{\linethickness}\end{center}

\hypertarget{libraries-used}{%
\section{Libraries Used}\label{libraries-used}}

\begin{verbatim}
citation("tidyverse")
citation("foreign")
citation("tidylog")
citation("janitor")
citation("jmv")
citation("tangram")
citation("finalfit")
citation("summarytools")
citation("ggstatplot")
citation("readxl")
\end{verbatim}

\begin{center}\rule{0.5\linewidth}{\linethickness}\end{center}

\begin{center}\rule{0.5\linewidth}{\linethickness}\end{center}

\begin{center}\rule{0.5\linewidth}{\linethickness}\end{center}

\hypertarget{notes}{%
\section{Notes}\label{notes}}

Completed on 2019-09-24 19:38:24.

Serdar Balci, MD, Pathologist\\
\href{mailto:drserdarbalci@gmail.com}{\nolinkurl{drserdarbalci@gmail.com}}\\
\url{https://rpubs.com/sbalci/CV}~\\
\url{https://sbalci.github.io/}~\\
\url{https://github.com/sbalci}

\begin{center}\rule{0.5\linewidth}{\linethickness}\end{center}


\end{document}
